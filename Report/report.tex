\documentclass[11pt]{article}

\usepackage[left=0.75in, right=0.75in, top=0.75in, bottom=0.75in]{geometry}
\usepackage{layout}
\usepackage{ucs}
\usepackage[french]{babel}
\usepackage[latin1]{inputenc}
\usepackage[T1]{fontenc}
\usepackage{titlesec}
\usepackage{graphicx}
\usepackage{amssymb}
\usepackage{amsmath}
\usepackage{dsfont}
\usepackage{caption}
\usepackage{subcaption}
\usepackage{array}
\usepackage{stmaryrd}



\title{\textbf{TS225 - Projet Signal}}
\author{Maxime PETERLIN - Gabriel VERMEULEN\\\\{ENSEIRB-MATMECA, Bordeaux}}
\date{14 novembre 2014}


\begin{document}

\maketitle
			
\tableofcontents

\newpage

\section{Analyse spectrale et pr�liminaire � l'analyse temps-fr�quence}

	\subsection{�tude th�orique}
	
		La FFT (\textit{Fast Fourier Transform}) est un algorithme de calcul de la transform�e de Fourier discr�te. Pour appliquer ce dernier, le signal que l'on cherche � �tudie doit �tre constitu� de $2^N$ points, avec $N \in \mathbb{N}$.
		Si ler signal analys� ne poss�de pas un nombre de points suffisant, on peut compl�ter ce dernier par des z�ros, ce proc�d� est nomm� zero-padding. Le nombre de point �tant augment� dans le domaine temporel, on gagne en pr�cision dans le domaine fr�quentiel.\\
		\\
		Matlab nous fourni de nombreux outils impl�mentant cette algorithme. On peut ainsi s'aider des fonctions suivantes :
			\begin{description}
				\item[fft] Permet de calculer une transform�e de Fourier discr�te avec l'algorithme de la transform�e de Fourier rapide ;
				\item[fft2] Permet de calculer une transform�e de Fourier 2D discr�te avec l'algorithme de la transform�e de Fourier rapide ;
				\item[fftshift] Permet de recentrer la transform�e de Fourier par rapport � la composante de fr�quence nulle.\\
			\end{description}
		\\
		\\
		$S(k) = A\cdot sin(2\pi \frac{f_0}{f_{ech}})$
		
	

\end{document}