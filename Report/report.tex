\documentclass[11pt]{article}

\usepackage[left=0.75in, right=0.75in, top=0.75in, bottom=0.75in]{geometry}
\usepackage{layout}
\usepackage{ucs}
\usepackage[french]{babel}
\usepackage[latin1]{inputenc}
\usepackage[T1]{fontenc}
\usepackage{titlesec}
\usepackage{graphicx}
\usepackage{amssymb}
\usepackage{amsmath}
\usepackage{dsfont}
\usepackage{caption}
\usepackage{subcaption}
\usepackage{array}
\usepackage{stmaryrd}
    
		\DeclareFontFamily{U}{wncy}{}
    \DeclareFontShape{U}{wncy}{m}{n}{<->wncyr10}{}
    \DeclareSymbolFont{mcy}{U}{wncy}{m}{n}
    \DeclareMathSymbol{\Sh}{\mathord}{mcy}{"58} 


\title{\textbf{TS225 - Projet Signal}}
\author{Maxime PETERLIN - Gabriel VERMEULEN\\\\{ENSEIRB-MATMECA, Bordeaux}}
\date{14 novembre 2014}


\begin{document}

\maketitle
			
\tableofcontents

\newpage

\section{Introduction}

\section{Analyse spectrale et pr�liminaire � l'analyse temps-fr�quence}

	\subsection{�tude th�orique}
	
		La FFT (\textit{Fast Fourier Transform}) est un algorithme de calcul de la transform�e de Fourier discr�te. Pour appliquer ce dernier, le signal que l'on cherche � �tudie doit �tre constitu� de $2^N$ points, avec $N \in \mathbb{N}$.
		Si le signal analys� ne poss�de pas un nombre de points suffisant, on peut compl�ter ce dernier par des z�ros, ce proc�d� est nomm� zero-padding. Le nombre de point �tant augment� dans le domaine temporel, on gagne en pr�cision dans le domaine fr�quentiel.\\
		\\
		Matlab nous fourni de nombreux outils impl�mentant cette algorithme. On peut ainsi s'aider des fonctions suivantes :
			\begin{description}
				\item[fft] qui permet de calculer une transform�e de Fourier discr�te avec l'algorithme de la transform�e de Fourier rapide ;
				\item[fft2] qui permet de calculer une transform�e de Fourier 2D discr�te avec l'algorithme de la transform�e de Fourier rapide ;
				\item[fftshift] qui permet de recentrer la transform�e de Fourier par rapport � la composante de fr�quence nulle.\\
			\end{description}
		\\
		\\
		Lorsqu'on �tudie des signaux, on ne peut le faire sur des temps inifinis. Ainsi, afin d'avoir un support temporel qui soit fini, on fen�tre les signaux �tudi�s. Nous allons, ici, analyser spectralement l'impact d'un tel fen�trage sur un signal sinuso�dal, que l'on �chantillonnera par la suite, repr�sent� par la fonction suivante : $s(x) = sin(2\pi f_0 t)$.\\
		On fen�tre ce signal rectangulairement par une porte de taille N, on obtient alors : 
		\[
			f(x) = sin(2\pi f_0 t) \cdot \Pi_{N}(t)
		\]
		Puis on �chantillonne ce dernier � une fr�quence $f_{ech}$ : 
		\[
			\tilde{f}(x) = sin(2\pi f_0 t) \cdot \Pi_{N}(t) \cdot \Sh_\frac{1}{f_{ech}}(t)
		\]
		\\
		On calcule alors la transform�e de Fourier de ce signal :
		\begin{align}
			\tilde{F}(\nu) &= \frac{1}{2j}[\delta(\nu+f_0) - \delta(\nu-f_0)] \otimes N\cdot sinc(N\nu) \otimes f_{ech} \Sh_{f_{ech}}(\nu) \\
			&= 
		\end{align}
		
		
	

\end{document}